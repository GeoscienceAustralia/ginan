\chapter{Installation}
\label{ch:installation}


In this chapter we will describe how to install the PEA and POD from source. An alternative option to installing all of the dependencies and the source code would be to use one of our docker images available from Docker Hub. Instructions on how to do this are in (see \hyperref[ch:docker]{Docker}).

\section{Supported Platforms}\label{supported-platforms}

Ginan is supported on the following platforms

\begin{itemize}
\item
  Linux
\item
  MacOS\\
\end{itemize}

\section{Download}\label{download}
You can download Ginan source from github using git clone:

\begin{lstlisting}[language=bash]
$ git clone https://github.com/GeoscienceAustralia/ginan.git
\end{lstlisting}

Then download all of the example data using the python script provided:

\begin{lstlisting}[language=bash]
$ python3 scripts/download_examples.py 
\end{lstlisting}


\section{Directory Structure}\label{directory-structure}

\begin{verbatim}
ginan/
├── README.md           ! General README information
├── LICENSE.md          ! Software License information
├── aws/                ! Amazon Web Services config
├── bin/                ! Binary executables directory*
├── CMakeLists.txt      ! Cmake build file
├── docs/               ! Documentation directory
├── examples/           ! Ginan examples directory
│   ├── data/           ! example dataset (rinex files)**
│   ├── products/       ! example products and aux files**
│   ├── solutions/      ! example solutions for QC**
│   --------------PEA examples--------------
│   ├── ex11            ! PEA example 1
│   ├── ex12            ! PEA example 2
│   ├── ex13            ! PEA example 3
│   ├── ex14            ! PEA example 4
│   ├── ex15            ! PEA example 5
│   ├── ex17            ! PEA example 7
│   ├── ex18            ! PEA example 8
│   --------------POD examples--------------
│   ├── ex21            ! POD example 1
│   ├── ex22            ! POD example 2
│   ├── ex23            ! POD example 3
│   ├── ex24            ! POD example 4
│   ├── ex25            ! POD example 5
│   └── ex26            ! POD example 6
│
├── lib/                ! Compiled objectlibrary directory*
├── scripts/            ! Auxillary Python and Shell scripts and libraries
└── src/                ! Source code directory
    ├── cpp/            ! PEA source code
    ├── fortran/        ! POD source code
    ├── cmake/   
    ├── doc_templates/
    ├── build/          ! Cmake build directory*
    └── CMakeLists.txt
\end{verbatim}

\emph{*created during installation process}

\emph{**created by \texttt{download\_examples.py} script}\\


\section{Dependencies}\label{dependencies}

Ginan has several software dependencies:

\begin{itemize}
\item
  C/C++ and Fortran compiler. We use and recommend
  \href{https://gcc.gnu.org}{gcc-g++ and gfortran}
\item
  BLAS and LAPACK linear algebra libraries. We use and recommend
  \href{https://www.openblas.net/}{OpenBlas} as this contains both
  libraries required
\item
  CMAKE \textgreater{} 3.0
\item
  YAML \textgreater{} 0.6
\item
  Boost \textgreater{} 1.70
\item
  Eigen3
\item
  netCDF4
\item
  Python3 (tested on Python 3.7)\\
\end{itemize}

\section{Installing dependencies with Ubuntu}\label{installing-dependencies-with-ubuntu}


Update the base operating system:

\begin{lstlisting}[language=bash]
$ sudo apt update
$ sudo apt upgrade
\end{lstlisting}

Install base utilities \texttt{gcc}, \texttt{gfortran}, \texttt{git},
\texttt{openssl}, \texttt{openblas} etc:

\begin{lstlisting}[language=bash]
$ sudo apt install -y git gobjc gobjc++ gfortran libopenblas-dev openssl curl net-tools openssh-server cmake make libssl1.0-dev
\end{lstlisting}


\section{Building additional
dependencies}\label{building-additional-dependencies}

Depending on the user's installation choice: install PEA-only, POD-only
or all software packages, a set of additional dependencies that need to
be built may change. Below, we explain building all the additional
dependencies:

First, create a temporary directory structure to make the dependencies
in, it can be removed after the installation process is done:

\begin{lstlisting}[language=bash]
$ sudo mkdir -p /data/tmp
$ cd /data/tmp
\end{lstlisting}

Note that \texttt{/data/tmp} is only used here as example and can be any
directory

\begin{itemize}
  

\item \textbf{YAML}
We are using the \href{https://github.com/jbeder/yaml-cpp}{YAML} library
to parse the configuration files used to run many of the programs found
in this library. Here is an example of how to install the yaml library
from source:

\begin{lstlisting}[language=bash]
$ cd /data/tmp
$ sudo git clone https://github.com/jbeder/yaml-cpp.git
$ cd yaml-cpp
$ sudo mkdir cmake-build
$ cd cmake-build
$ sudo cmake .. -DCMAKE\_INSTALL\_PREFIX=/usr/local/ -DYAML\_CPP\_BUILD\_TESTS=OFF
$ sudo make install yaml-cpp
$ cd ../..
$ sudo rm -fr yaml-cpp
\end{lstlisting}

\item\textbf{Boost (PEA)}

PEA relies on a number of the utilities provided by
\href{https://www.boost.org/}{boost}, such as their time and logging
libraries.

\begin{lstlisting}[language=bash]
$ cd /data/tmp/
$ sudo wget -c https://boostorg.jfrog.io/artifactory/main/release/1.73.0/source/boost_1_73_0.tar.gz
$ sudo gunzip boost_1_73_0.tar.gz
$ sudo tar xvf boost_1_73_0.tar
$ cd boost_1_73_0/
$ sudo ./bootstrap.sh
$ sudo ./b2 install
$ cd ..
$ sudo rm -fr boost_1_73_0/ boost_1_73_0.tar
\end{lstlisting}

\item\textbf{Eigen3 (PEA)}

Eigen3 is used for performing matrix calculations in PEA, and has a very
nice API.

\begin{lstlisting}[language=bash]
$ cd /data/tmp/
$ sudo git clone https://gitlab.com/libeigen/eigen.git
$ cd eigen
$ sudo mkdir cmake-build
$ cd cmake-build
$ sudo cmake ..
$ sudo make install
$ cd ../..
$ sudo rm -rf eigen
\end{lstlisting}

\item\textbf{MongoDB (PEA, optional)}

Needed for realtime preview of the processed results (developers-only)

\begin{lstlisting}[language=bash]
$ wget https://github.com/mongodb/mongo-c-driver/releases/download/1.17.1/mongo-c-driver-1.17.1.tar.gz
$ tar -xvf mongo-c-driver-1.17.1.tar.gz

$ cd mongo-c-driver-1.17.1/
$ mkdir cmakebuild
$ cd cmakebuild/
$ cmake -DENABLE_AUTOMATIC_INIT_AND_CLEANUP=OFF ..
$ cmake --build .
$ sudo cmake --build . --target install

$ cd ../../
$ curl -OL https://github.com/mongodb/mongo-cxx-driver/releases/download/r3.6.0/mongo-cxx-driver-r3.6.0.tar.gz
$ tar -xzf mongo-cxx-driver-r3.6.0.tar.gz

$ cd mongo-cxx-driver-r3.6.0/

$ cd build/
$ cmake -DCMAKE_BUILD_TYPE=Release -DCMAKE_INSTALL_PREFIX=/usr/local ..
$ sudo cmake --build . --target EP_mnmlstc_core
$ cmake --build .
$ sudo cmake --build . --target install

$ wget -qO - https://www.mongodb.org/static/pgp/server-4.4.asc | sudo apt-key add -
$ echo "deb [ arch=amd64,arm64 ] https://repo.mongodb.org/apt/ubuntu focal/mongodb-org/4.4 multiverse" | sudo tee /etc/apt/sources.list.d/mongodb-org-4.4.list
$ echo "deb [ arch=amd64,arm64 ] https://repo.mongodb.org/apt/ubuntu bionic/mongodb-org/4.4 multiverse" | sudo tee /etc/apt/sources.list.d/mongodb-org-4.4.list

$ sudo apt update
$ sudo apt install mongodb-org 

$ sudo systemctl start mongod
$ sudo systemctl status mongod
$ mongod
\end{lstlisting}

\item\textbf{netcdf4 (OTL package)}

\begin{lstlisting}[language=bash]
$ sudo apt -y install libnetcdf-dev libnetcdf-c++4-dev
\end{lstlisting}

\end{itemize}

\section{Build}\label{build}

Prepare a directory to build in, its better practise to keep this
seperated from the source code.

\begin{lstlisting}[language=bash]
$ cd src
$ mkdir -p build
$ cd build
\end{lstlisting}

Run cmake to find the build dependencies and create the make file. If
you wish to enable the optional MONGO DB utilities you will need to add
the \texttt{-DENABLE\_MONGODB=TRUE} flag. If you wish to compile an
optimised version, typically this version will run 3 times faster but
you may run into compile problems depending on your system, add the
\texttt{-DOPTIMISATION=TRUE} flag:

\begin{lstlisting}[language=bash]
$ cmake [-DENABLE_MONGODB=TRUE] [-DENABLE_OPTIMISATION=TRUE] ..
\end{lstlisting}

To build every package simply run \texttt{make} or \texttt{make\ -j\ 2},
where 2 is a number of parallel threads you want to use for the
compilation:

\begin{lstlisting}[language=bash]
$ make [-j 2]
\end{lstlisting}

To build specific package (e.g.~PEA or POD), run as below:

\begin{lstlisting}[language=bash]
$ make pea -j 2
$ make pod -j 2
\end{lstlisting}

This should create executables in the \texttt{bin} directory of Ginan.

Check to see if you can execute the PEA:

\begin{lstlisting}[language=bash]
$ ../../bin/pea --help
\end{lstlisting}

and you should see something similar to:

\begin{lstlisting}[language=bash]
PEA starting... (pea_pod_examples vbf8c9cc from Tue Jul 6 06:09:50 2021)
Options:
--help                      Help
--quiet                     Less output
--verbose                   More output
--very-verbose              Much more output
--config arg                Configuration file
--trace_level arg           Trace level
--antenna arg               ANTEX file
--navigation arg            Navigation file
--sinex arg                 SINEX file
--sp3files arg              Orbit (SP3) file
--clkfiles arg              Clock (CLK) file
--dcbfiles arg              Code Bias (DCB) file
--bsxfiles arg              Bias Sinex (BSX) file
--ionfiles arg              Ionosphere (IONEX) file
--podfiles arg              Orbits (POD) file
--blqfiles arg              BLQ (Ocean loading) file
--erpfiles arg              ERP file
--elevation_mask arg        Elevation Mask
--max_epochs arg            Maximum Epochs
--epoch_interval arg        Epoch Interval
--rnx arg                   RINEX station file
--root_input_dir arg        Directory containg the input data
--root_output_directory arg Output directory
--start_epoch arg           Start date/time
--end_epoch arg             Stop date/time
--run_rts_only arg          RTS filename (without _xxxxx suffix)
--dump-config-only          Dump the configuration and exit
--input_persistance         Begin with previously stored filter and 
                            navigation states
--output_persistance        Store filter and navigation states for restarting
PEA finished
\end{lstlisting}

Similarly, check the POD:

\begin{lstlisting}[language=bash]
$ ../../bin/pod --help
\end{lstlisting}

This returns:

\begin{lstlisting}[language=bash]
Earth Radiation Model (ERM):   1

Default master POD config file = POD.in
To run from default config file: ../../bin/pod or ../../bin/pod -c POD.in

POD.in config file options by defaut can be overridden on the command line

Command line: ../../bin/pod -m -s -o -a -p -r -t -n -i -u -q -k -w -y -h 

Where: 
    -m --podmode = POD Mode:
                                1 - Orbit Determination (pseudo-observations; orbit fitting)
                                2 - Orbit Determination and Prediction
                                3 - Orbit Integration (Equation of Motion only)
                                4 - Orbit Integration and Partials (Equation of Motion and Variational Equations)
    -s --pobs    = Pseudo observations orbit .sp3 file name
    -o --cobs    = Comparison orbit .sp3 file name
    -a --arclen  = Orbit Estimation Arc length (hours)
    -p --predlen = Orbit Prediction Arc length (hours)
    -r --eopf    = Earth Orientation Paramaeter (EOP) values file
    -t --eopsol  = Earth Orientation Paramaeter file type: (1,2,3)
                                1 - IERS C04 EOP
                                2 - IERS RS/PC Daily EOP
                                3 - IGS RP + IERS RS/PC Daily (dX,dY)
    -n --nutpre  = IAU Precession / Nutation model
                                2000 - IAU2000A
                                2006 - IAU2006/2000A
    -i --estiter = Orbit Estimatimation Iterations (1 or greater)
    -u --sp3vel  = Output .sp3 file with velocities
                                0 - Do not write Velocity vector to sp3 orbit
                                1 - Write Velocity vector to sp3 orbit
    -q --icmode  = Initial condition from parameter estimation procedure
    -k --srpmodel= 1: ECOM1, 2:ECOM2, 12:ECOM12, 3:SBOX
    -w --empmodel= 1: activated, 0: no estimation
    -d --verbosity = output verbosity level [Default: 0]
    -y --yaml = yaml config file
    -h --help.   = Print program help

Examples:

        ../../bin/pod -m 1 -q 1 -k 1 -w 0 -s igs16403.sp3 -o igs16403.sp3 -y ex1.yaml
        ../../bin/pod -m 2 -q 1 -k 1 -w 0 -s igs16403.sp3 -p 12 -y ex2.yaml

For orbit updates using Parameter Estimation Algorithm (PEA):
        ../../bin/pod -m 4 -q 2 -k 1 -w 0 -s igs16403.sp3 -o igs16403.sp3 -y ex3.yaml
\end{lstlisting}


\section{Documentation}\label{documentation}

Ginan documentation consists of two parts: a doxygen-generated
documentation that shows the actual code infrastructure and a detailed
manual, written in latex, that provides an overview of the software, a
thoretical background, a detailed ``how to'' guide etc.\\
Below, we explain on how to generate each bit of documentation:

\subsection{Doxygen}\label{doxygen}

The Doxygen documentation for Ginan requires \texttt{doxygen} and
\texttt{graphviz}. If not already installed, type as follows:

\begin{lstlisting}[language=bash]
$ sudo apt -y install doxygen graphviz
\end{lstlisting}

On success, proceed to the build directory and call make with
\texttt{doc\_doxygen} target:

\begin{lstlisting}[language=bash]
$ cd src/build
$ make doc_doxygen
\end{lstlisting}

The docs can then be found at \texttt{docs/html/index.html}. Note that
documentation is generated automatically if \texttt{make} is called
without arguments and \texttt{doxygen} and \texttt{graphviz}
dependencies are satisfied.

\subsection{Latex}\label{latex}

A detailed Ginan manual is located in \texttt{docs/manual} and is in
latex format. To compile Latex to pdf you will need a compiler, such as
texlive:

\begin{lstlisting}[language=bash]
$ sudo apt install texlive-latex-base texlive-latex-extra
\end{lstlisting}

Now, go to \texttt{docs/manual} and generate the pdf:

\begin{lstlisting}[language=bash]
$ cd docs/manual
$ pdflatex main.tex
$ makeglossaries main
$ pdflatex main.tex
\end{lstlisting}

\texttt{main.pdf} file should now appear in the directory.\\


\section{Ready!}\label{ready}

Congratulations! You are now ready to trial the examples of PEA and POD
from the examples directory. See Ginan's manual for detailed explanation
of each example. Note that examples have relative paths to files in them
and rely on the presence of \texttt{products}, \texttt{data} and
\texttt{solutions} directories inside the \texttt{examples} directory.
Make sure you've run \texttt{download\_examples.py} from the
\hyperref[download]{download} step of this instruction.

The paths are relative to the examples directory and hence all the
examples must be run from the \texttt{examples} directory.

\begin{lstlisting}[language=bash]
cd ../../examples
\end{lstlisting}

To run the first example of the PEA:

\begin{lstlisting}[language=bash]
../bin/pea --config ex11_pea_pp_user_gps.yaml
\end{lstlisting}

This should create \texttt{ex11} directory with
\texttt{ex11-ALIC201919900.TRACE} and \texttt{ex1120624.snx} output
files. You can remove the need for path specification to the executable
by adding Ginan's bin directory to \texttt{\textasciitilde{}/.bachrc}
file:

\begin{lstlisting}[language=bash]
PATH="path_to_ginan_bin:$PATH"
\end{lstlisting}

And an example of POD:

\begin{lstlisting}[language=bash]
../bin/pod -y ex21_pod_fit_gps.yaml
\end{lstlisting}

At the completion of the test run, \texttt{ex21} directory should be
create. The \texttt{ex21\_.sh} script will return any differences to the
standard test resuts.


\section{Python Installation for Plotting, Processing,
etc.}\label{python-installation-for-plotting-processing-etc.}

Lastly, to run many of the included scripts for fast parsing of
.trace/.snx files, plotting of results, automatic running of the PEA
based on input date/times and stations, etc. then a number of python
dependencies are needed.

The file scripts/conda\_gn37.yaml has a list of the necessary python
dependencies.\\
The best way to take advantage of this is to install the Miniconda
virtual environment manager.\\
This will allow you to pass the .yaml file into the conda command and
automatically set up a new python environment.

\begin{enumerate}
  \item\textbf{Install Miniconda}

To install Miniconda, download and execute the Miniconda shell file:

\begin{lstlisting}[language=bash]
$ wget https://repo.anaconda.com/miniconda/Miniconda3-latest-Linux-x86_64.sh
$ bash Miniconda3-latest-Linux-x86_64.sh
\end{lstlisting}

And follow the on-screen instructions (choosing all defaults is fine).

\item\textbf{Create virtual
environment}

After installation you can create the \texttt{gn37} python environment
using a prepared receipy. First open a new terminal session and enter:

\begin{lstlisting}[language=bash]
$ conda env create -f <dir_to_ginan>/scripts/conda_gn37.yaml
\end{lstlisting}

\end{enumerate}

You have now created the virtual python environment \texttt{gn37} with
all necessary dependencies. Anytime you wish you run python scripts,
ensure you are in the virtual environment by activating:

\begin{lstlisting}[language=bash]
$ conda activate gn37
\end{lstlisting}

And then run your desired script from the \texttt{scripts} directory.